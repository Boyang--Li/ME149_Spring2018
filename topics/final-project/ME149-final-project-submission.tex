% This file contains the header data for all assignment files
\documentclass[onecolumn, 11pt]{article}

\usepackage{bbold}   %Get fancy double struck math notation for sets
\usepackage{cite}
\usepackage{cleveref}
\usepackage{color}
\usepackage{courier}   %Have code written out nicely
\usepackage{float}
\usepackage[top=1in, bottom=1in, left=1in, right=1in]{geometry}
\usepackage{graphicx} % handles graphics and figures
\usepackage{hyperref}
\usepackage{listings}
\usepackage{mathtools,bm}
\usepackage{multicol}

\catcode`\^^M=10      %  Makes blank lines meaningless, force use of \par

\definecolor{magenta}{rgb}{0.8,0.0,1.0}
\definecolor{darkGreen}{rgb}{0.0,0.4,0.0}
\definecolor{blue}{rgb}{0.0, 0.0, 0.9}
\definecolor{purple}{rgb}{0.7, 0.0, 0.7}
\definecolor{darkGreen}{rgb}{0.0,0.4,0.0}

\newcommand{\quotes}[1]{``#1''}
\newcommand{\todo}[1]{{\color{magenta}\par {[{\bf ToDo: } {\em #1}} ] \\    }}

\newcommand{\norm}[1]{\left\lVert#1\right\rVert}

%%%%%%%%%%%%%%%%%%%%%%%%%%%%%%%%%%%%%%%%%%%%%%%%%%%%%%%%%%%%%%%%%%%%%%%%%%%%%%%
% NOTE:
%
% The following block of commands is used to format Matlab code blocks for the
% listings package. This block of code is based on two examples:
%  -->   https://gist.github.com/eyliu/120689
%  -->   http://links.tedpavlic.com/ascii/homework_new_tex.ascii
%
% Import file block using:
%   \lstinputlisting{fileName.m}
%
\lstloadlanguages{Matlab}
%
\lstset{language=Matlab,                        % Use MATLAB
        frame=single,                           % Single frame around code
        basicstyle=\small\ttfamily,             % Use small true type font
        keywordstyle=[1]\color{blue}\bfseries,  % MATLAB functions bold and blue
        keywordstyle=[2]\color{purple},         % MATLAB function arguments purple
        keywordstyle=[3]\color{blue}\underbar,  % User functions underlined and blue
        identifierstyle=,                       % Nothing special about identifiers
                                                % Comments small dark green courier
        commentstyle=\usefont{T1}{pcr}{m}{sl}\color{darkGreen}\small,
        stringstyle=\color{darkGreen},            % Strings are purple
        showstringspaces=false,                 % Don't put marks in string spaces
        tabsize=4,                              % 4 spaces per tab
        %
        %%% Put standard MATLAB functions not included in the default
        %%% language here
        morekeywords={xlim,ylim,var,alpha,factorial,poissrnd,normpdf,normcdf},
        %
        %%% Put MATLAB function parameters here
        morekeywords=[2]{on, off, interp},
        %
        %%% Put user defined functions here
        morekeywords=[3]{FindESS, homework_example},
        %
        morecomment=[l][\color{blue}]{...},     % Line continuation (...) like blue comment
        numbers=left,                           % Line numbers on left
        firstnumber=1,                          % Line numbers start with line 1
        numberstyle=\tiny\color{blue},          % Line numbers are blue
        stepnumber=5                            % Line numbers go in steps of 5
        }
%
%%%%%%%%%%%%%%%%%%%%%%%%%%%%%%%%%%%%%%%%%%%%%%%%%%%%%%%%%%%%%%%%%%%%%%%%%%%%%%%


%========================================================================
\title{ME 149:  Final Project Submission}
\date{Assigned: April 3  ---  Due: April 29 at 11:55pm}
\author{Optimal Control for Robotics}
%========================================================================
\begin{document}
\maketitle

%=================================================

\section*{Deliverables}

You should submit two items to Trunk for your final project.
The first is a \texttt{.pdf} format write-up and
the second is a single \texttt{.zip} file that contains all of your source code.
Both files are described in detail in the following sections.

\par
If you are working with a partner, then both students should upload report and
source code files to trunk.
The files should be identical and the report should include both students names.

%~~~~~~~~~~~~~~~~~~~~~~~~~~~~~~~~~~~~~~~~~~~~~~~~~~~~~~~~~~~~~~~~~~~~~~~~~~~~~

\section*{Project Report}

The main body of the report should not exceed four pages.
You may choose to include an appendix.
The appendix has no page limit but each section must be
referenced from within the main body of the report.
The report should be written in a clear and concise manner, and it should
contain equations, figures, and references where appropriate.
The report can be organized however is best for your project,
but take a look a some conference papers for inspiration.
A good report will include
a clear description of the problems being solved, written both in english and equations;
a clear statement of the transcription method(s) and the resulting non-linear program; and
a summary of the results.

%~~~~~~~~~~~~~~~~~~~~~~~~~~~~~~~~~~~~~~~~~~~~~~~~~~~~~~~~~~~~~~~~~~~~~~~~~~~~~

\section*{Source Code}

The source code should be well written and clearly documented.
There are several requirements:

\vspace{-0.0em} \begin{itemize}  \setlength\itemsep{0em} \setlength\itemindent{4pt}

  \item There must be a \texttt{README.md} file in the top-level directory.
        As the name suggests, this file should be written in Markdown.
        This file should provide an outline of how your code is organized
        and any other information that will be useful to someone reading
        your source code. This file should also include a description
        for each entry-point file that you have provided.

  \item There should be one or more entry-point scripts in your top-level
        directory. These scripts should be named \texttt{MAIN\_<description>.m},
        and each should include documentation at the top of the file that
        describes what it does in detail.

  \item The entry-point scripts should run! You may assume that the test
        computer will have the code library for the course on the Matlab path.
        Clearly state any other dependencies in the \texttt{README.md} file
        as well as the entry-point scripts.

  \item Your function files should all be named in \texttt{camelCase.md} and
        provide help documentation following the format used throughout the
        class in the assignment solutions.

  \item You code should be well written and well documented.

  \item Running each entry-point script should produce a set of plots that
        provide useful information about the problem being solved and the solution.

  \item Create a directory called \texttt{results} and save a \texttt{.pdf} version
        of each figure that your code generates.

\end{itemize}

%~~~~~~~~~~~~~~~~~~~~~~~~~~~~~~~~~~~~~~~~~~~~~~~~~~~~~~~~~~~~~~~~~~~~~~~~~~~~~

\section*{Grading}

I have included a draft of the rubric that I plan to use for grading the
final project. This is included as a guide for helping to focus your work on
the project, rather than as a strict contract.
I will award up to five additional points for projects that are ambitious and well done.

\vspace{-0.0em} \begin{itemize}  \setlength\itemsep{0em} \setlength\itemindent{4pt}

  \item \textbf{Meets Basic Requirements: 10pts}
  \item \textbf{Meets Advanced Requirements: 10pts}
  \item \textbf{Code Runs: 10pts}
  \item \textbf{\textttt{README.md}: 5 pts}
  \item \textbf{Code --- Documentation: 5 pts}
  \item \textbf{Code --- Style: 5pts}
  \item \textbf{Code --- Correct Implementation: 10pts}
  \item \textbf{Report --- Problem Statement: 10}
  \item \textbf{Report --- Methods / Transcription: 10}
  \item \textbf{Report --- Results: 10}
  \item \textbf{Report --- Correct Math: 5 pts}
  \item \textbf{Report --- Clarity and Organization: 10 pts}


\end{itemize}
%=================================================
\end{document}
